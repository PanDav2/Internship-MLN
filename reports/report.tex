%%%%%%%%%%%class file
\documentclass{imamci}


%this command gives the journal no. 
%
\jno{dnnxxx}
%%%%%%%%%%%%%%


%%%%%%%%%%%%%%%%%
%call  packages
\usepackage{natbib}
%%%%%%%%%%%%%%

%%%%%%%%%%%%%%%%%%

\usepackage{graphicx}
\usepackage{amsmath,amsthm,bm,mathrsfs}

%%%%%%%%%%%%%%%%%


%%%%%%%%%%%
%this files contains Theorem styles based in IMA JOURNALS
%
\input standard.tex

%%%%%%%%%%%%%


\renewcommand{\theequation}{\thesection.\arabic{equation}}
\numberwithin{equation}{section}
\def\citeasnoun{\cite}



\begin{document}

%%%%%%%%%%%%%%%%%
\title{Article summarization}
\author{ {\sc David A.PANOU}\\[2pt]
School of Mathematics, University of Manchester, \\
 Sackville Street, Manchester, M60 1QD, UK.\\[6pt]
{\rm [Received on 18 February 2005]}\vspace*{6pt}}
\pagestyle{headings}
\markboth{C. E. POWELL}{\rm PARAMETER-FREE H(DIV) PRECONDITIONING}
\maketitle

%%%%%%%%%%%%%%%%%abstract style
%Two grouping braces are necessary in abstract environment
%first argument contains abstract text; second argument contains keywords
%text

\begin{abstract}
{
Des différentes applications des Markov Logic Network

Mixed finite element formulations of generalised diffusion problems yield linear systems with  ill-conditioned, symmetric and indefinite coefficient matrices. Preconditioners with optimal work complexity that do not rely on artificial parameters are essential. We implement lowest-order Raviart-Thomas elements and analyse practical issues associated with so-called $`H({\it
div})$ preconditioning'. Properties of the exact scheme are discussed in  Powell \& Silvester (2003), `Optimal preconditioning for Raviart-Thomas mixed formulation of second-order elliptic problems,' \textit{SIAM J. Matrix Anal. Appl.,} \textbf{25}(3), 718--738. We extend the discussion, here, to practical implementation, the components of which are any available multilevel solver for a weighted $H({\it div})$ operator and a pressure mass matrix. A new bound is established for the eigenvalue spectrum of the preconditioned system matrix and extensive numerical results are presented.}
{saddle point problems, mixed finite elements, Raviart-Thomas, preconditioning.}
\end{abstract}
%%%%%%%%%%%%%%%%%%%%%%%%%%%%%%%



%%%%%%%%%%%%%%section A%%%%%%%%%
\section{Introduction}

%%%%%%%%%%%%%%%%%%%%%%%%%%%%%

Let $\Omega$ be a polygon in $I\!\!R^{d} $ $(d=2,3)$ and consider the boundary value-problem,
\begin{align}\label{mixedcont}
\begin{split}
\mathcal{A}^{-1}\vec{u}-\nabla p &=  0,\\
\nabla \cdot \vec {u} &= -f \quad  \mbox{in } \Omega,\\
p &= g
\quad\ \ \ \, \mbox{on } \partial\Omega_{D},\\
\vec {u}\cdot\vec {n} &= 0
\quad\quad \mbox{on } \partial\Omega_{N},
\end{split}
\end{align}
where $\partial\Omega_{D}\neq \emptyset$ and $\mathcal{A} = \mathcal{A}(\vec{x})$
is a $d \times d$ bounded, symmetric and uniformly positive definite matrix-valued
function with minimum eigenvalue bounded away from zero. 

%%%%%%%%%%%Section B
\subsection{Second level section}
%%%%%%%%%%%%%%%%%%%

The mixed first-order system (\ref{mixedcont}) occurs in models of fluid
flow in porous media (see Russell \& Wheeler (1983) and Ewing {\it et al.}
(1983).) The macroscopic flow of groundwater satisfies $\vec{u} =   -k\mu^{-1} \nabla\break
P_{R},$ where $\vec{u}$ denotes fluid discharge, $P_{R}$ is `residual pressure',
$k$ is the permeability coefficient and $\mu$ is viscosity. Coupling Darcy's
law with mass conservation yields (\ref{mixedcont}) with $f=0$, $\mathcal{A}
= -\frac{k}{\mu}\mathcal{I}$ and $p:=P_{R}$. To fix ideas, we call $p$ and
$\vec{u}= \mathcal{A} \nabla p$ the `pressure' and `velocity' solutions,
respectively.  Flow domains are often comprised of different media with spatially
varying permeability coefficients, leading to heterogeneous problems with
discontinuous $\mathcal{A}.$ In stratified media, the entries of $\mathcal{A}$
corresponding to different co-ordinate directions vary in magnitude, yielding
anisotropic $\mathcal{A}.$ Mixed finite element methods are favoured when
$\vec{u}$ is the variable of interest as post-processing primal solutions
leads to loss of accuracy. Low order mixed methods also conserve mass locally.

The Krylov subspace method \textsc{minres} (see Paige \& Saunder (1975)) is an
optimal solver for (\ref{RT_matrix}). The $k$th iterate minimises the Euclidean
norm of the $k$th residual error over the corresponding Krylov space and,
since $C$ is sparse, the cost per iteration depends linearly on the problem
size. Popular alternative soltution schemes include Uzawa's method and the
augmented Lagrangian method (see Fortin \& Glowinski (1983, Ch. 1)). However, these
methods require nested iteration and the user's choice of relaxation parameter
determines the convergence of the outer iteration. Deficiencies associated
with this are highlighted in Rusten \& Winther (1992). Fast convergence is obtained
for the outer iteration if the relaxation parameter is tuned in the right
way but for the augmented Lagrangian method, that cripples the inner-iteration.
Preconditioners for the inner-solve have been suggested, see Hiptmair (1997),
but optimal parameter values have not been discussed. Vassilevski \& Lazarov
(1996)
use \textsc{minres} but  introduce artificial parameters. Applying \textsc{minres}
directly to (\ref{RT_matrix}) is simpler and more user-friendly.
%%%%%%%%%%%%%%this is lemma style
\begin{lemma}\label{identitybd}
The $(n+m)$ eigenvalues of the generalised eigenvalue problem,
\begin{equation}
\left( \begin{array}{@{\ }cc@{\ }}
\mathnormal{A}_{I} & \mathnormal{B}^{T} \\
\mathnormal{B} & 0
\end{array} \right) \left( \begin{array}{@{\ }c@{\ }} 
 \underline{u} \\ 
 \underline{p}
\end{array} \right) = \sigma\left(\begin{array}{@{\ }cc@{\ }}
\mathnormal{A}_{I} + \mathnormal{D} & 0 \\
0 & \mathnormal{N} 
\end{array}\right) \left(\begin{array}{@{\ }c@{\ }} 
 \underline{u} \\
 \underline{p}
\end{array}\right),\label{RT_matrix}
\end{equation}
arising in the Raviart-Thomas approximation of (\ref{neweq}) are bounded by
constants independent of $h$ and lie in the intervals $\left[ -1, -\beta_{*}^{2}
\right] \cup \left[ 1 \right],$ where $\beta_{*}$ is the discrete inf-sup
constant in (\ref{neweq}).
\end{lemma}
%%%%%%%%%%%%%%%%%%%%%%%%%%%%%%
\begin{proof}
It remains to prove the last assertion. The idea is to take ${\xi}$ equal to ${\rm grad}\,\theta$ in the first line of (2.10). This yields
$$
\int_\Omega \vert {\hbox{grad\,}}\theta\vert^2\,{\rm d} {\textbf{\textit{x}}} = 0.
$$
So, since $\Omega$ is connected, $\theta$ is a constant and its nullity follows
from the boundary conditions that it satisfies on $\Gamma_0$.
\end{proof}


\noindent When $\mathcal{A}=\mathcal{I},$ choosing $\mathnormal{P}_{1}$ and
$\mathnormal{P}_{2}$ to represent the norms for which stability holds, leads
to an $h$-optimal eigenvalue bound. For problems with general coefficient
tensors, however, $\mathnormal{P}$ must also supply scaling with respect
to $\mathcal{A}$. We choose $P_{1} = A+D,$ representing the weighted norm
$\| \cdot \|_{{\it div}, \mathcal{A}},$ induced by the inner-product (\ref{Vconstants}).

\section{Practical $H({\it div})$ preconditioning} 

Now, let $\mathnormal{V}$ be \textit{any} symmetric and positive
definite approximation to $\mathnormal{H}=A+D.$ An ideal choice is a V-cycle
of multigrid, the cost of which is known to depend linearly on the problem
size (see Trottenberg {\it et al.} (2001, p. 74).) However, \textit{any} available approximation
can be substituted, provided that there exist positive constants $\theta$
and $\Theta,$ satisfying,
\begin{eqnarray}\label{Vconstants}
0 < \theta \le \frac{\underline{u}^{T}\mathnormal{H}\underline{u}}{\underline{u}^{T}
\mathnormal{V}\underline{u}} \le  \Theta \le  1  \quad \forall\, \underline{u} \in I\!\!R^{n}\backslash\{\underline{0}\}. 
\end{eqnarray}
The condition $\Theta \le 1$ is not restrictive, since the chosen $\mathnormal{V}$
can always be rescaled. It is purely to simplify presentation in the sequel.
In section 5 we perform numerical computations with a particular $V$ and
compute corresponding values of $\theta$ and $\Theta.$ 

Now consider the preconditioner,
\begin{equation}\label{pracprec}
P= \left(\begin{array}{@{\ }cc@{\ }}
\mathnormal{V}& 0 \\
0 & \mathnormal{N} 
\end{array}\right).
\end{equation}
We require bounds for the eigenvalues of $P^{-1}C.$ To simplify notation
further, let\footnote{In the context of $\mathscr{H}$-matrices, this twofold hierarchy is called $\mathscr{H}^{2}$-format.}
\begin{equation}\label{exacta}
a = \left(\frac{c\mu_{min}}{\mid T \mid_{min}  + \mu_{min}}\right), 
\end{equation}
so that the ideal bound (\ref{pracprec}) reads $[-1, -a] \cup [1]$. In the
proof, given in Powell \& Silvester (2003, Lemma 2.3), we establish that the negative
eigenvalues of (\ref{exacta}) are the values $\left\{ \lambda \right\}$
satisfying, $BH^{-1}\break B^{T}\underline{p} = - \lambda N \underline{p}.$ In other
words,
\begin{eqnarray}\label{newadded}
 0 < a  \le  \frac{\underline{p}^{T}BH^{-1}B^{T}\underline{p}}{\underline{p}^{T}N
\underline{p}} \le  1 \quad \forall \, \underline{p} \in I\!\!R^{m}\backslash\{\underline{0}\}.
\end{eqnarray}

%%%%%%%%%%%%%%this is theorem style
\begin{theorem}\label{bigresult}
The $(n+m)$ eigenvalues $\{\lambda_{i}\}_{i=1}^{n+m}$ of the generalised
eigenvalue problem,
\begin{equation}\label{approxprec} 
\left( \begin{array}{@{\ }cc@{\ }}
\mathnormal{A} & \mathnormal{B}^{T} \\
\mathnormal{B} & 0
\end{array} \right) \left( \begin{array}{@{\ }c@{\ }} 
 \underline{u} \\
 \underline{p} 
\end{array} \right) = \lambda\left(\begin{array}{@{\ }cc@{\ }}
 \mathnormal{V}& 0 \\
 0 & \mathnormal{N} 
 \end{array}\right) \left(\begin{array}{@{\ }c@{\ }} 
 \underline{u} \\ 
 \underline{p}
\end{array}\right),
\end{equation}
in the Raviart-Thomas approximation of (\ref{approxprec}), lie in the union of the
intervals, 
\begin{equation}\label{bigsimple} 
\left[ -1,\frac{1}{2}\left(\theta\left(1-a\right)-\sqrt{\theta^{2}\left(a
-1\right)^{2}+4a\theta}\right) \right] \cup  [\theta, 1], 
\end{equation}
where $\theta$ is the constant satisfying (\ref{Vconstants}) and $a$ is the
constant defined in (\ref{exacta}).
\end{theorem}
%%%%%%%%%%%%%%%%%%%%%%%
%
%
%%%%%%%%% proof style
\begin{proof}
First note that since the eigenvalues $\{\lambda_{i}\}_{i=1}^{m+n}$ satisfy,
 \[
\mathnormal{A}\underline{u} + \mathnormal{B}^{T}\underline{p}
= \lambda V\underline{u},\quad \quad\mathnormal{B}\underline{u} = \lambda
\mathnormal{N}\underline{p},
 \]
eliminating $\underline{p}$ yields $\lambda \underline{u}^{T}A\underline{u}
+ \underline{u}^{T}D\underline{u} = \lambda^{2}\underline{u}^{T}V\underline{u}$.
Applying \eqref{Vconstants} yields,
\begin{eqnarray}
\label{neweq} \lambda^{2}\underline{u}^{T}H\underline{u} \le \lambda 
\underline{u}^{T}A\underline{u} + \underline{u}^{T}D\underline{u}  \le  
\lambda^{2}\theta^{-1}\underline{u}^{T}H\underline{u}.
\end{eqnarray}
From the left inequality we obtain,
\begin{eqnarray}
\label{added} \left(\lambda^{2} - \lambda\right)\underline{u}^{T}A\underline{u}
+ \left(\lambda^{2} - 1 \right)\underline{u}^{T}D\underline{u} \le  0,
\end{eqnarray}
and since $A$ is positive definite and $D$ is semi-positive definite, we
immediately establish $|\lambda| \le 1.$ The right inequality in \eqref{neweq}
yields,
 \[
\left(1 - \lambda\right) \underline{u}^{T} D
\underline{u}  \le  \left(\lambda^{2}\theta^{-1} - \lambda
\right) \underline{u}^{T} H \underline{u}.
 \]
Since $0 \le 1-\lambda $ it follows that $0  \le  \lambda \left(\lambda\theta^{-1}
- 1 \right)$. Hence if $\lambda >0$, we have $\lambda \ge \theta$, and the
bound for the positive eigenvalues is proved. 
\begin{BL}
\item $\overline{\Omega} = \bigcup_{T \in\mathcal{M}} \overline T$;

\item for any $(T,T') \in \mathcal{M}^2$, $T \not =  T'$ $ \Longrightarrow T \cap T' = \emptyset$;  

\item for any $(T,T') \in \mathcal{M}^2$, $\overline T \cap \overline T' = \emptyset$ or $\overline T \cap \overline T'$ is a common edge (or a face in three-space dimensions) or a common vertex of $T$ and $T'$.
\end{BL}
\begin{definition}
Let 
\begin{equation}
\sum_{j=0}^{k}\alpha_{j}u^{n+j-k}={\Delta t}\sum_{j=0}^{k}%
\beta_{j}f(u^{n+j-k})  \label{eqn:4}
\end{equation}
be a linear multistep method for an ordinary differential equation %
\mbox{$u^{\prime}(t)=f(u(t))$}, where $u^{n}\approx u(t_{n})$. Define 
\begin{equation*}
\gamma(\zeta):=\frac{\sum_{j=0}^{k}\alpha_{j}\zeta^{k-j}}{%
\sum_{j=0}^{k}\beta_{j}\zeta^{k-j}}
\end{equation*}
as the quotient of its generating polynomials. %
\end{definition}
\begin{algorithm}
Since $u$ is also the solution to (2.1), from (2.2)
we have that $b(v,p)=0$ for all $v\in X$. Thanks to the inf--sup condition (2.1),
the result follows immediately.
\end{algorithm}



Now assume that $\lambda <0$. Eliminating $\underline{u}$ yields,
\begin{equation}
\label{negeigs} B\left(\lambda V-A\right)^{-1}B^{T}
\underline{p} = \lambda N \underline{p}.
\end{equation}
The values of $\lambda$ satisfying (\ref{negeigs})
are the eigenvalues of the matrix,
\begin{align*}
\mathnormal{N}^{-\frac{1}{2}}\mathnormal{B}\left(\lambda
V - A \right)^{-1}\mathnormal{B}^{T}\mathnormal{N}^{-\frac{1}{2}}
& = \mathnormal{N}^{-\frac{1}{2}}\mathnormal{B}\left(\lambda V -
H
+ D\right)^{-1}\mathnormal{B}^{T}\mathnormal{N}^{-\frac{1}{2}} \\
 & =\mathnormal{N}^{-\frac{1}{2}}\mathnormal{B}\left(\lambda V - H
+ B^{T}N^{-1}B\right)^{-1}\mathnormal{B}^{T}\mathnormal{N}^{-\frac{1}{2}}\\
 &= \mathnormal{N}^{-\frac{1}{2}}\mathnormal{B}Y^{-\frac{1}{2}}\left(I
+ Y^{-\frac{1}{2}} B^{T}N^{-1}B Y^{-\frac{1}{2}}\right)^{-1} Y^{-\frac{1}{2}}
 \mathnormal{B}^{T}\mathnormal{N}^{-\frac{1}{2}} \\
 & =\mathnormal{X}\left(\mathnormal{I} + \mathnormal{X}^{T}\mathnormal{X}
 \right)^{-1}\mathnormal{X}^{T},
\end{align*}
where, here, $\mathnormal{X} = \mathnormal{N}^{-\frac{1}{2}}\mathnormal{B}
\mathnormal{Y}^{-\frac{1}{2}}$ and $Y=\lambda V - H$. Applying the 
Sherman-Morrison-Woodbury formula to $\left(I + XX^{T} \right)^{-1}$ yields,
\begin{equation}
\label{SMW}
X\left(I + XX^{T} \right)^{-1}X^{T}= X\left(I-X^{T}\left(I + XX^{T} \right)^{-1}
X\right)X^{T}.
\end{equation}
Let $\underline{v}$ be an eigenvector of $XX^{T}$ and let $\sigma$ denote
the corresponding eigenvalue. Then, with (\ref{SMW}), we obtain,
\begin{align*}
 X\left(I + XX^{T} \right)^{-1}X^{T}\underline{v} & =   XX^{T}\underline{v}
 - XX^{T}\left(I + XX^{T} \right)^{-1}XX^{T}\underline{v} \\
 & =  \sigma \underline{v} - \left(\frac{\sigma^{2}}{1+\sigma}\right) \underline{v}
 =  \left( \frac{\sigma}{1 + \sigma}\right) \underline{v}.
\end{align*}
Hence, the values $\{\lambda\}$ we are seeking in (\ref{negeigs}) satisfy,
\begin{equation}\label{sl} 
\lambda_{i} = \frac{\sigma_{i}}{1 +\sigma_{i}}, 
\end{equation} 
where each $\sigma_{i}$ is an eigenvalue of,
\begin{equation}
\label{needthis} B\left(\lambda_{i}V - H
\right)^{-1}B^{T}\underline{p} = \sigma N \underline{p}.
\end{equation}
We now can obtain a bound for these values by exploiting the spectral equivalence
of $H$ and $V$ defined by (\ref{Vconstants}). (Note that we have no readily
available information about the spectral equivalence of the leading blocks
$A$ and $V$.)
\end{proof}

Consider, first, the eigenvalues $\{ \mu \}$ of,
\begin{eqnarray}
\label{star2} \left(\lambda V - H \right)^{-1} \underline{u} =
\mu  H^{-1} \underline{u}.
\end{eqnarray}
Since $\left(\lambda V - H\right)$ is negative definite and $H^{-1}$ is positive
definite, the values of $\mu$ are negative. Rearranging \eqref{star2} and
applying (\ref{Vconstants}) yields $\theta \le \lambda \mu \left(\mu + 1\right)^{-1}
\le  1$. Recalling that $\theta >0$, $\mu<0$ and $\lambda<0$, we find that,
\begin{equation}
\label{star3} \mu  \in \left[ \frac{1}{\lambda - 1},
\frac{\theta}{\lambda - \theta} \right].
\end{equation}

%%%%%%%%%remark style
\begin{remark}
When $\theta = 1$, we recover the eigenvalue bound (\ref{eigs1}).
\end{remark}

\begin{corollary}
\label{Corfdee}Let the assumptions of Theorem 2.1 be satisfied.
Let $\Delta t\sim h^{m+3/2}$ and choose $\varepsilon\sim h^{7m/2+25/4}$.
Then, the solution $\tilde{\phi}_{\Delta t,h}$ exists and converges with
the optimal rate
\begin{equation*}
\|\tilde{\phi}_{\Delta t,h}^{n}-\phi(\cdot,t_{n})
\|_{H^{-1/2}(\Gamma)}\leq C_{g}(T)
h^{m+3/2}\sim C_{g}(T) \Delta t^{2}.
\end{equation*}
\end{corollary}

%%%%%%%%%%%%%%%%%%%%%




%%%%%%%%%%%%%%This is table style%%%%
%This \tblcaption styles contains 2 argument
%first argument is caption style; second argument is table body style
%\tblhead defines table heading style
%\lastline defines last line of the table 
\begin{table}[!b] %T1
\tblcaption{Eigenvalues of  $V^{-1}H$, unit coefficients}
{\mbox{\tabcolsep=18pt\begin{tabular}{@{}cccc@{}}
\tblhead{$h$ & $\frac{1}{4}$  & $\frac{1}{8}$ &$\frac{1}{16}$\\[-9.5pt]}\\[-9.5pt]
$\theta$ & 0.5938 & 0.4595 & 0.4273    \\
$\Theta$ &  1 & 1 &1 \lastline
\end{tabular}
\label{eigs1}
}}
\end{table}
%%%%%%%%%%%%%%%%%%%%%%%%%%%%%%%%%%%%%


%%%%%%%%%%%%This is table style with notes
%
\begin{table}[b!] %T2
\tblcaptionnotes{Theoretical bounds and observed eigenvalues,
unit coefficients}
{\mbox{\tabcolsep=10pt\begin{tabular}{@{}ccc@{}}
\tblhead{$h$ & $bounds^{a}$ &$observed^{b}$\\[-11pt]}\\[-9.5pt]
$\frac{1}{4}$   &$[-0.9983,\, -0.7381]  \cup  [0.5938 ,1]$ &
$[-0.9879, \, -0.8507] \cup [0.5943 ,1] $\\[3pt]
$\frac{1}{8}$   &  $[-0.9996, \, -0.6504] \cup [0.4595 ,1]$
&$[-0.9972, \, -0.8438] \cup [0.4598 ,1]$\\[3pt]
$\frac{1}{16}$  &  $[-0.9999, \, -0.6503] \cup [0.4273 ,1]$ &
$[-0.9994, \, -0.8481] \cup [0.4273 ,1] $\\[-9pt] \lastline
\end{tabular}
\label{Neigs2}
}}{$^a$ Iteration counts obtained with the ideal preconditioner are
listed in columns 2--3 of Table 2\\$^b$ The negative eigenvalues
of the preconditioned saddle point system.}\vspace*{-2pt}
\end{table}


%%%%%%%%%%%%%%%%%example style
\begin{example}
Next, introduce a jump in the coefficient and set $\mathcal{A} = \alpha \mathcal{I}$
in one quadrant of $\Omega$ so that $\mu_{min} \to 0$ in (\ref{exacta})
as $\alpha \to 0$ (see Powell \& Silvester (2003).) Values of $\theta$ are listed
in Table \ref{eigs1};  $\Theta =1$ in all cases. The approximation to $\mathcal{H
}_{\mathcal{A}}$ is $\mathcal{A}$-optimal and $h$-optimal. The negative eigenvalues
of the preconditioned saddle point system, for $\alpha = 10^{-3}$ and $10^{-6}$
are listed in Tables~\ref{eigs1}--\ref{Neigs2}. Observe that the right-hand
bound for the negative eigenvalues is tighter as $\alpha \to 0.$ The eigenvalues
of the preconditioned saddle point system, for fixed $h$ and varying $\alpha$
are plotted in Fig. \ref{approxeig}. The scale on the $y$-axis corresponds
to values of $\alpha \in \left[10^{-6}, 10^{6}\right]$ and each line of the
plot depicts the eigenvalues for a different value of $\alpha.$ 


 Clearly, for small values $\alpha < <1$, \textsc{minres} convergence will
deteriorate. Iteration counts obtained with the ideal preconditioner are
listed in columns 2--4 of Table 1. Counts for the multigrid
version are given in columns 5--7. As Theorem \ref{bigresult} predicts, the
multigrid preconditioner exhibits exactly the same asymptotic behaviour as
the ideal version as $\alpha \to 0$. 



Clearly, for small values $\alpha < <1$, \textsc{minres} convergence will
deteriorate. Iteration counts obtained with the ideal preconditioner are
listed in columns 2--4 of Table 1. Counts for the multigrid
version are given in columns 5--7. As Theorem \ref{bigresult} predicts, the
multigrid preconditioner exhibits exactly the same asymptotic behaviour as
the ideal version as $\alpha \to 0$. Note that the deterioration in convergence
has nothing to do with the chosen multigrid solver. It performs optimally,
with respect to $h$ and $\mathcal{A}.$ In this simple case, the deterioration
can be corrected by rescaling $\mathcal{A}$ so that the minimum value of
any coefficient is unity.
\end{example}
\begin{case}
  The operator $M$ is clearly bounded; let us prove that it also has closed
  range. To see this, use the existence of a bounded extension
  operator from $H^2(\Omega)$ to $H^2(T^2)$ to observe that the range of
  $M$ is $(M)=\{(\phi,\gamma)^T\in \phi_{|\Omega^c}=0\}$. By the continuity of the restriction operator we have that this set is closed.
  
  An operator between two Hilbert spaces which is bounded and has
  closed range has a bounded Moore--Penrose pseudoinverse (see, e.g. Arnold
  {\it et al.}, 1997). Thus, the pseudoinverse $M^\dagger$ of $M$ exists
  and is bounded.
\end{case}


%%%%%%%%%%%%%%%%%%%%%%%%%%%%%%%%%


\markboth{C. E. POWELL}{\rm PARAMETER-FREE H(DIV) PRECONDITIONING}


\section{Concluding remarks} 
Motivated by standard stability theory and discussion in Arnold {\it et al.}
(2000) and Powell \& Silvester (2003), we described a block-diagonal parameter free preconditioning
scheme for the linear systems (\ref{RT_matrix}) arising in the lowest-order
Raviart-Thomas discretisation of generalised diffusion problems. New bounds
are established for the eigenvalue spectrum of the preconditioned saddle
point matrix when the leading block of the ideal preconditioner is replaced
with a suitable approximation. In numerical experiments, we demonstrated
the impact of general coefficient tensors on the performance of a particular
multilevel approximation and on the theoretical eigenvalue bound.
%%%%%%%%%%%%%figure style
\begin{figure}[!t] %F1
\centering\includegraphics[width=0.45\textwidth] {sisceigs2}
\caption{Eigenvalues of preconditioned saddle point system,
$h=\frac{1}{16}, \alpha \in \left[10^{-6}, 10^{6}\right]$ }
\label{approxeig}\vspace*{-9pt}
\end{figure}
%%%%%%%%%%%%%%%%%%%%%%%%%%%%%%



\vspace*{6pt}

\begin{references}\markboth{C. E. POWELL}{\rm PARAMETER-FREE H(DIV) PRECONDITIONING}

\item{}
\textsc{Arnold, D.N., Falk, R.S. \& Winther, R.} (1997)   Preconditioning
in $H({\it div})$ and applications,   {\em Math. Comp.,}   \textbf{66}
(219), 957--984.

\item{}
\textsc{Arnold, D.N., Falk, R.S. \& Winther, R.} (2000)   Multigrid
in $H({\it div})$ and $H(curl)$, {\em Numer. Math.,}   \textbf{85},\break 197--218.

\item{}
\textsc{Brezzi, F.} (1974)  {On the existence, uniqueness and approximation
of saddle point problems arising from\break Lagrangian multipliers,}
 {\em RAIRO Anal. Num\'er.,} \textbf{8}, 129--151.

\item{}
\textsc{Brezzi, F. \& Bathe, K.J.} (1990)  {A discourse on the stability
conditions for mixed finite element formulations},  {\em Comp. Methods.
Appl. Mech. Engrg.,}   \textbf{82}, 27--57.

\item{}
\textsc{Brezzi, F. \& Fortin, M.} (1991)  {\em Mixed and Hybrid Finite
Element Methods.}   New York: Springer-Verlag.

\item{}
\textsc{Cai, Z., Goldstein, C.I. \& Pasciak, J.E.} (1993)  {Multilevel
iteration for mixed finite element systems with penalty},  {\em SIAM
J. Sci. Comput.,}   \textbf{14} (5), 1072--1088.

\item{}
\textsc{Ewing, R.E. \& Wheeler, M.F.} (1983)  {Computational aspects
of mixed finite element methods.}  {In: \textsc{R. Stepleman,} ed.
 \em Numerical methods for Scientific Computing,} North-Holland, 163--172.

\item{}
\textsc{Ewing, R.E. \& Wang, J.} (1992)  {Analysis of the Schwarz
algorithm for mixed finite elements methods},  {\em M$^2$AN, Math.
Model. Numer. Anal.,}   \textbf{26} (6), 739--756.

\item{}
\textsc{Ewing, R.E. \& Wang, J.} (1994)  {Analysis of multilevel
decomposition iterative methods for mixed finite element methods},  {\em M$^2$AN, Math. Model. Numer. Anal.,}   \textbf{28} (4), 377--398.


\item{}
\textsc{Fortin, M. \& Glowinski, R.} (1983)  {\em Augmented Lagrangian
Methods: Applications to the Numerical Solution of Boundary-Value Problems.}
Amsterdam: North-Holland.

\item{}
\textsc{Hiptmair, R.} (1997)  {Multigrid methods for $H({\it div})$ in
three dimensions},  { \em Electron. Trans. Numer. Anal.,}  
\textbf{6}, 133--152.

\item{}
\textsc{Kellogg, R.B.,} (1975)  {On the Poisson equation with intersecting
interfaces},  {\em Appl., Anal.,}   \textbf{4}, 101--129.

\item{}
\textsc{Morin, P., Nochetto, R.H. \& Siebert, K.G.} (2002)  {Convergence
of adaptive finite element methods},  {\em SIAM Rev.,}  
1--28.

\item{}
\textsc{Paige, C.C. \& Saunders, M.A.} (1975)  {Solution of sparse
indefinite systems of linear equations,}  {\em SIAM J. Numer. Anal.,}
  \textbf{12}, 617--629.

\item{}
\textsc{Powell, C.E. \& Silvester, D.} (2003)  {Optimal preconditioning
for Raviart-Thomas mixed formulation of second-order elliptic problems,}
 {\em SIAM J. Matrix Anal. Appl.,} \textbf{25} (3), 718--738.

\item{}
\textsc{Raviart, P.A. \& Thomas, J.M.} (1977)  {A mixed finite element
method for second order elliptic problems.}  {In: \textsc{I. Galligani
and E. Magenes,} eds. \em Mathematical Aspects of the Finite Element Method,
Lect. Notes in Math.,} 606, New York: Springer-Verlag, 292--315.

\item{}
\textsc{Russell, T.F. \& Wheeler, M.F.} (1983)  {Finite element and
finite difference methods for continuous flows in porous media.}  {In:
\textsc{R.E. Ewing,} ed. \em  The Mathematics of Reservoir Simulation.} Philadelphia:
SIAM, 35--106.

\item{}
\textsc{Rusten, T. \& Winther, R.} (1992)  {A preconditioned iterative
method for saddlepoint problems},  {\em SIAM J. Matrix. Anal. Appl.,}
  \textbf{13} (3), 887--904.

\item{}
\textsc{Rusten, T., Vassilevski, P.S. \& Winther, R.} (1996)  {Interior
preconditioners for mixed finite element approximations of elliptic problems},
 {\em Math. Comp.,}   \textbf{65} (214), 447--466.

\item{}
\textsc{Scheichl, R.} (2000)  {\em Iterative solution of saddle point
problems using divergence-free finite elements with applications to groundwater
flow.} Thesis (PhD). Bath University.

\item{}
\textsc{Trottenberg, U., Oosterlee, C. \& Sch\"uller, A.} (2001)  {Multigrid.}
 {Academic Press.}

\item{}
\textsc{Vassilevski, P.S. \& Wang, J.} (1992)  {Multilevel iterative
methods for mixed finite element discretizations of elliptic problems},  {\em
Numer. Math.},   \textbf{63}, 503--520.

\item{}
\textsc{Vassilevski, P.S. \& Lazarov, R.D.} (1996)  {Preconditioning
mixed finite element saddle point elliptic problems,}  {\em Numer.
Linear Algebra Appl.,}   \textbf{3} (1), 1--20.
\end{references}



\end{document}


%%%%%%%%%%%%%%%%%bibliography style
\begin{thebibliography}{99}\markboth{C. E. POWELL}{\rm PARAMETER-FREE H(DIV) PRECONDITIONING}
\harvarditem{Arnold \textit{et al.}}{1997}{arnoldthree}
\textsc{Arnold, D.N., Falk, R.S. \& Winther, R.} (1997)   Preconditioning
in $H({\it div})$ and applications,   {\em Math. Comp.,}   \textbf{66}
(219), 957--984.

\harvarditem{Arnold \textit{et al.}}{2000}{arnoldtwo}
\textsc{Arnold, D.N., Falk, R.S. \& Winther, R.} (2000)   Multigrid
in $H({\it div})$ and $H(curl)$,   {\em Numer. Math.,}   \textbf{85},
197--218.

\harvarditem{Brezzi}{1974}{brezzi1}
\textsc{Brezzi, F.} (1974)  {On the existence, uniqueness and approximation
of saddle point problems arising from Lagrangian multipliers,}
 {\em RAIRO Anal. Num\'er.,} \textbf{8}, 129--151.

\harvarditem{Brezzi \& Bathe}{1990}{brezzibathe}
\textsc{Brezzi, F. \& Bathe, K.J.} (1990)  {A discourse on the stability
conditions for mixed finite element formulations},  {\em Comp. Methods.
Appl. Mech. Engrg.,}   \textbf{82}, 27--57.

\harvarditem{Brezzi \& Fortin}{1991}{brezzifortin}
\textsc{Brezzi, F. \& Fortin, M.} (1991)  {\em Mixed and Hybrid Finite
Element Methods.}   New York: Springer-Verlag.

\harvarditem{Cai \textit{et al.}}{1993}{cai2}
\textsc{Cai, Z., Goldstein, C.I. \& Pasciak, J.E.} (1993)  {Multilevel
iteration for mixed finite element systems with penalty},  {\em SIAM
J. Sci. Comput.,}   \textbf{14} (5), 1072--1088.

\harvarditem{Ewing \textit{et al.}}{1983}{ewingwheeler}
\textsc{Ewing, R.E. \& Wheeler, M.F.} (1983)  {Computational aspects
of mixed finite element methods.}  {In: \textsc{R. Stepleman,} ed.
 \em Numerical methods for Scientific Computing,} North-Holland, 163--172.

\harvarditem{Ewing \& Wang}{1992}{ewingone}
\textsc{Ewing, R.E. \& Wang, J.} (1992)  {Analysis of the Schwarz
algorithm for mixed finite elements methods},  {\em M$^2$AN, Math.
Model. Numer. Anal.,}   \textbf{26} (6), 739--756.

\harvarditem{Ewing \& Wang}{1994} {ewingtwo}
\textsc{Ewing, R.E. \& Wang, J.} (1994)  {Analysis of multilevel
decomposition iterative methods for mixed finite element methods},  {\em M$^2$AN, Math. Model. Numer. Anal.,}   \textbf{28} (4), 377--398.


\harvarditem{Fortin \& Glowinski}{1983}{glow}
\textsc{Fortin, M. \& Glowinski, R.} (1983)  {\em Augmented Lagrangian
Methods: Applications to the Numerical Solution of Boundary-Value Problems.}
Amsterdam: North-Holland.

\harvarditem{Hiptmair}{1997}{hiptmair}
\textsc{Hiptmair, R.} (1997)  {Multigrid methods for $H({\it div})$ in
three dimensions},  { \em Electron. Trans. Numer. Anal.,}  
\textbf{6}, 133--152.

\harvarditem{Kellogg}{1975}{kellogg}
\textsc{Kellogg, R.B.,} (1975)  {On the Poisson equation with intersecting
interfaces},  {\em Appl., Anal.,}   \textbf{4}, 101--129.

\harvarditem{Morin \textit{et al.}}{2002}{morin}
\textsc{Morin, P., Nochetto, R.H. \& Siebert, K.G.} (2002)  {Convergence
of adaptive finite element methods},  {\em SIAM Rev.,}  
1--28.

\harvarditem{Paige \& Saunders}{1975}{paige}
\textsc{Paige, C.C. \& Saunders, M.A.} (1975)  {Solution of sparse
indefinite systems of linear equations,}  {\em SIAM J. Numer. Anal.,}
  \textbf{12}, 617--629.

\harvarditem{Powell \& Silvester}{2003}{me1}
\textsc{Powell, C.E. \& Silvester, D.} (2003)  {Optimal preconditioning
for Raviart-Thomas mixed formulation of second-order elliptic problems,}
 {\em SIAM J. Matrix Anal. Appl.,} \textbf{25} (3), 718--738.

\harvarditem{Raviart \& Thomas}{1977}{ravtom}
\textsc{Raviart, P.A. \& Thomas, J.M.} (1977)  {A mixed finite element
method for second order elliptic problems.}  {In: \textsc{I. Galligani
and E. Magenes,} eds. \em Mathematical Aspects of the Finite Element Method,
Lect. Notes in Math.,} 606, New York: Springer-Verlag, 292--315.

\harvarditem{Russell \& Wheeler}{1983}{russell}
\textsc{Russell, T.F. \& Wheeler, M.F.} (1983)  {Finite element and
finite difference methods for continuous flows in porous media.}  {In:
\textsc{R.E. Ewing,} ed. \em  The Mathematics of Reservoir Simulation.} Philadelphia:
SIAM, 35--106.

\harvarditem{Rusten \& Winther}{1992}{rusten}
\textsc{Rusten, T. \& Winther, R.} (1992)  {A preconditioned iterative
method for saddlepoint problems},  {\em SIAM J. Matrix. Anal. Appl.,}
  \textbf{13} (3), 887--904.

\harvarditem{Rusten \textit{et al.}}{1996}{rusten3}
\textsc{Rusten, T., Vassilevski, P.S. \& Winther, R.} (1996)  {Interior
preconditioners for mixed finite element approximations of elliptic problems},
 {\em Math. Comp.,}   \textbf{65} (214), 447--466.

\harvarditem{Scheichl}{2000}{scheichl}
\textsc{Scheichl, R.} (2000)  {\em Iterative solution of saddle point
problems using divergence-free finite elements with applications to groundwater
flow.} Thesis (PhD). Bath University.

\harvarditem{Trottenberg \textit{et al.}}{2001}{trot}
\textsc{Trottenberg, U., Oosterlee, C. \& Sch\"uller, A.} (2001)  {Multigrid.}
 {Academic Press.}

\harvarditem{Vassilevski \& Wang}{1992}{vasswang}
\textsc{Vassilevski, P.S. \& Wang, J.} (1992)  {Multilevel iterative
methods for mixed finite element discretizations of elliptic problems},  {\em
Numer. Math.},   \textbf{63}, 503--520.

\harvarditem{Vassilevski \& Lazarov}{1996}{vasslaz}
\textsc{Vassilevski, P.S. \& Lazarov, R.D.} (1996)  {Preconditioning
mixed finite element saddle point elliptic problems,}  {\em Numer.
Linear Algebra Appl.,}   \textbf{3} (1), 1--20.
\end{thebibliography}
%%%%%%%%%%%%%%%%%%%%%%%%
%%%%%%%%%%%%%%%%%%%%%%%%%%%%%%%%%%%

\end{document}
%%%%%%%%%%%%%%% this is title page style%%%%%%%%%
